\section{Conclusion and Future Work}
\label{sec:conclusion}

This paper presented an edge-driven system for CO\textsubscript{2} and occupancy prediction aimed at optimizing classroom ventilation. By integrating real-time IoT sensor data (CO\textsubscript{2}, temperature, humidity, occupancy) with an LSTM predictive model deployed on a Raspberry Pi (fog layer) controlling Arduino-based edge nodes, the system demonstrated effective environmental monitoring and proactive ventilation management. Key achievements include the successful implementation of the hierarchical edge-fog-cloud architecture, accurate prediction of environmental trends enabling calculations of KPIv and Room Quality Score, and the automated generation of environmental quality certificates via ThingSpeak integration.

The results validate the feasibility of using lightweight LSTM models on edge/fog devices for real-time analysis and control in resource-constrained environments like classrooms. The system effectively bridges low-cost IoT deployment with intelligent building management principles, offering a data-backed approach to ensuring healthier indoor air quality. The ability to provide verifiable environmental quality certificates adds a layer of accountability and transparency.

Future work will focus on enhancing the system's practicality and scalability. Key directions include:
\begin{itemize}
    \item \textbf{Wireless Communication:} Replacing the current wired serial connection between edge and fog layers with wireless protocols like Zigbee or LoRaWAN to improve deployment flexibility.
    \item \textbf{Model Enhancement:} Training the LSTM model on larger, more diverse datasets incorporating data from various room types and conditions to improve generalizability and accuracy.
    \item \textbf{Expanded Sensing:} Incorporating additional sensors for parameters like Particulate Matter (PM2.5) and Volatile Organic Compounds (VOCs) to provide a more comprehensive IAQ assessment.
    \item \textbf{HVAC Integration:} Interfacing the system directly with building HVAC systems for fully automated and optimized ventilation control, moving beyond simple fan actuation.
    \item \textbf{Advanced ML Techniques:} Exploring techniques like federated learning for enhanced privacy and adaptive learning for real-time model fine-tuning directly on edge/fog devices.
    \item \textbf{Scalability Testing:} Deploying the system across a larger number of classrooms or zones to evaluate performance and management at scale.
\end{itemize}
By pursuing these enhancements, the system can evolve into a more robust, scalable, and intelligent solution, contributing significantly to the development of healthier, energy-efficient, and data-driven educational environments. 