\section{Introduction}
\label{sec:introduction}

Indoor air quality (IAQ) in educational settings significantly impacts student cognitive function, comfort, and overall health, particularly in densely occupied classrooms \cite{Azuma2018}. Traditional ventilation systems often rely on static thresholds or manual controls, proving inefficient and unresponsive to dynamic changes in occupancy and environmental conditions like Carbon Dioxide (CO\textsubscript{2}) levels \cite{McNeill2022}. Such systems struggle to adapt to fluctuating occupancy, varying room sizes, and diverse usage patterns, leading to inconsistent air quality, potential health concerns, and suboptimal energy consumption.

The primary challenge lies in developing an intelligent, responsive ventilation system capable of real-time data acquisition, accurate prediction of environmental changes, and automated control actions. While cloud-based solutions exist, they often suffer from latency issues unsuitable for immediate environmental adjustments \cite{Idrees2021}. Edge computing, coupled with advancements in Internet of Things (IoT) sensors and deep learning models like Long Short-Term Memory (LSTM) networks, offers a promising alternative by enabling localized data processing and predictive control with minimal delay.

This paper proposes an edge-driven system for real-time CO\textsubscript{2} and occupancy prediction to optimize classroom ventilation. Our approach integrates low-cost IoT sensors for environmental data collection (temperature, humidity, CO\textsubscript{2}, occupancy), a lightweight LSTM model deployed on edge devices for predictive analysis, and certificate generation for environmental quality assurance. The system employs a hierarchical architecture, leveraging edge devices for immediate processing and a fog layer for aggregation, alongside cloud integration for long-term data storage and analysis.

The key contributions of this work are:
\begin{itemize}
    \item Development and edge deployment of a lightweight LSTM model for real-time prediction of CO\textsubscript{2} levels and occupancy patterns in classrooms.
    \item Design and implementation of a hierarchical edge-fog-cloud architecture for efficient data processing, low-latency control, and scalability.
    \item Real-time generation of environmental quality certificates based on continuous monitoring and prediction.
    \item Demonstration of the system's effectiveness in maintaining optimal classroom conditions through adaptive ventilation adjustments.
\end{itemize}

The remainder of this paper is organized as follows: Section \ref{sec:related_work} reviews related work in smart ventilation and edge computing. Section \ref{sec:methodology} details the proposed system architecture and methodology. Section \ref{sec:results} presents the experimental results and evaluation. Finally, Section \ref{sec:conclusion} concludes the paper and discusses future work. 