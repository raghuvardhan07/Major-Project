\chapter{Literature Survey}

\section{Related Work}  
The growing concern over indoor air quality (IAQ) and energy efficiency has led to extensive research into smart ventilation systems using Internet of Things (IoT) technologies. Early solutions primarily utilized simple sensors with rule-based actuation, which lacked adaptability and real-time optimization. However, the integration of IoT and edge computing has enabled real-time sensing, data-driven control, and intelligent decision-making in ventilation systems. Researchers have developed systems that not only monitor pollutants like CO₂ and PM2.5 but also predict future air quality trends using machine learning models. Studies such as Luo et al. \cite{1} and Rastogi et al.\cite{2} have explored natural ventilation and adaptive ventilation control through IoT-based sensor data. Additionally, embedded systems like those proposed by Mahbub et al. \cite{3} have enabled multifunctional monitoring — including ventilation, lighting, and fire detection highlighting the potential of integrated smart building technologies.

\section{Current State of the Field}  
Contemporary smart ventilation systems utilize embedded platforms (such as ESP32 or Arduino) connected to multi-sensor modules (e.g., MQ-135, DHT11) to continuously monitor air quality indicators. These systems are often linked to cloud or edge-based platforms for real-time control and data visualization. As seen in Tagliabue et al.\cite{4} and Wang et al. \cite{5}, data-driven models such as LSTM and hybrid learning architectures are being used to predict pollutant levels and dynamically adjust ventilation systems. There is also a shift toward deploying models at the edge to reduce latency and dependence on centralized infrastructure. Smart dashboards, cloud APIs, and mobile applications facilitate user interaction and enhance usability. Despite technological progress, challenges such as sensor drift, environmental variability, and energy management still require robust solutions
\section{Recent Developments, Breakthroughs, and Trends}  
Recent advancements in smart ventilation systems research have focused on several key areas. One major development is the integration of edge AI, where machine learning models are deployed directly on edge devices for fast, localized decision-making. This approach enhances system responsiveness, allowing for real-time adjustments in ventilation. Another important direction is the development of multi-room and distributed systems, which optimize air quality across multiple indoor zones simultaneously. These systems enable more coordinated control of indoor environments, improving overall air quality management. Additionally, energy-aware ventilation systems have gained attention, with research emphasizing the balance between energy optimization and air quality enhancement. These systems are aligned with sustainable building goals, focusing on reducing energy consumption while maintaining indoor comfort. Furthermore, data-driven prediction models, particularly Long Short-Term Memory (LSTM) networks and hybrid methods, have been increasingly used for long-term prediction of air quality parameters. These models offer more accurate forecasting capabilities, enhancing the efficiency of ventilation systems. Lastly, low-cost modular IoT platforms have become a significant area of focus, enabling scalable and cost-efficient indoor air quality (IAQ) systems. This trend makes it feasible to implement smart ventilation technologies in various settings, including residential and commercial spaces, thus promoting broader adoption of sustainable indoor air quality management systems.

\section{Key Papers, Researchers, and Organizations}  

\subsection{Key Papers}  
Several influential papers have contributed significantly to the advancement of ventilation systems and IoT-based environmental monitoring. Maohui Luo et al. (2021) demonstrated the use of real-time sensor data to assess the ventilation potential in dynamic environments, highlighting the importance of IoT-based air quality sensors for efficient building ventilation \cite{1}. Rastogi et al. (2021) proposed a context-aware ventilation control system that adapts to indoor environmental parameters, ensuring optimized air quality and energy efficiency \cite{2}. Mahbub et al. (2021) introduced a multi-functional embedded system for indoor ventilation and fire detection, showcasing the integration of IoT in environmental monitoring and safety \cite{3}. Tagliabue et al. (2021) presented an advanced data-driven model for indoor air quality prediction in educational facilities, utilizing a sensor network to enhance IAQ management \cite{4}. Yang et al. (2021) developed a hybrid learning model, BO–EMD–LSTM, for reliable long-term CO₂ concentration forecasting, contributing to accurate predictions for ventilation needs \cite{6}. McNeill et al. (2022) explored IAQ optimization strategies in schools and universities, focusing on real-world applications of room-level ventilation control \cite{7}. Zivelonghi (2023) proposed a smart framework for dynamic, room-specific ventilation control, using IoT to improve air quality across multiple zones in educational environments \cite{11}. Benammar et al. (2018) introduced a low-cost, scalable modular IoT platform for real-time IAQ monitoring, making it feasible to deploy such systems in a variety of indoor settings \cite{16}. El-Sayed et al. (2017) highlighted the integration of edge computing with IoT and cloud technologies for smart environments, emphasizing the role of distributed computing in improving ventilation system performance \cite{14}.




\subsection{Prominent Researchers}  
Several researchers have made notable contributions to the development of IoT-based indoor air quality monitoring and smart ventilation systems. Maohui Luo is widely recognized for his pioneering work in assessing the potential for natural ventilation in buildings using real-time sensor data. His research demonstrated how IoT-based sensor networks can optimize ventilation in dynamic indoor environments, improving both air quality and energy efficiency in diverse building types. K. Rastogi has significantly advanced the field by developing context-aware systems that monitor and control ventilation rates based on changing indoor conditions. His work on adaptive ventilation control using IoT systems has allowed for more efficient management of indoor air quality, which is crucial in maintaining healthy environments in homes, offices, and public spaces. Mobasshir Mahbub’s contributions include the development of cloud-enabled, embedded IoT systems that integrate ventilation control with fire detection and lighting management. His research highlights the importance of multifunctional systems that offer a comprehensive approach to environmental monitoring while improving safety and comfort in indoor spaces. Lavinia Chiara Tagliabue’s work has focused on data-driven approaches to IAQ prediction, particularly in educational settings. She introduced advanced prediction models based on sensor networks to forecast air quality in classrooms, ensuring that ventilation systems could be optimized to enhance comfort and learning conditions. Guangfei Yang’s innovative approach involved hybrid machine learning models, such as BO-EMD-LSTM, for predicting long-term CO₂ concentrations in indoor spaces. His research has enabled more accurate long-term forecasting of air quality, which is essential for maintaining optimal conditions in classrooms and other indoor environments. V. Faye McNeill has worked extensively on optimizing room-level ventilation in schools and universities. Her research focuses on strategies that can improve air quality in individual rooms, offering practical solutions for improving IAQ in real-world academic institutions. Zivelonghi Giuseppi has also made significant strides in dynamic, room-specific air quality control, developing IoT-enabled frameworks that enable real-time air quality optimization across multiple rooms in schools. Mohieddine Benammar contributed to the development of modular IoT platforms for real-time indoor air quality monitoring, providing scalable solutions that can be deployed across various building types. Finally, Hesham El-Sayed’s research in integrating edge computing with IoT and cloud systems has transformed smart ventilation systems by enabling faster processing and real-time decision-making, reducing latency and improving system responsiveness.


\section{Literature Review}  
Recently, a variety of research efforts have focused on exploring advancements in IoT-driven ventilation systems designed to improve indoor air quality, safety, and energy conservation. With the advent of IoT-enabled sensor networks and data-driven frameworks, real-time indoor air quality monitoring and management have become achievable, laying the groundwork for more intelligent ventilation solutions. This survey reviews existing literature on IoT-based ventilation technologies, focusing on sensor innovations, data collection methodologies, machine learning applications, and control strategies that optimize both air quality and energy use.

Luo et al. (2018-2019) explored the potential of natural ventilation (NV) in buildings by deploying IoT-driven environmental sensors in a Berkeley building to monitor indoor air quality, climate, and occupant behavior over a year-long period. Their research showed that NV had the potential to be utilized for 61.2\% of the year, but its actual use was significantly lower, under 35\%. This discrepancy was largely attributed to occupants' preferences, which were not always aligned with the optimal environmental conditions. Occupants often preferred mechanical ventilation or had individual preferences that did not always match the ideal energy-saving strategies provided by natural ventilation. The study suggests that predictive models, which dynamically adjust ventilation strategies based on real-time environmental conditions and occupant behavior, could bridge this gap. Moreover, Luo et al. emphasized the importance of raising user awareness and leveraging advanced predictive models, which could lead to a more widespread adoption of NV, ensuring both improved indoor air quality and energy efficiency \cite{1}.

Rastogi et al. proposed a context-aware IoT-based system for monitoring indoor ventilation by integrating environmental sensors that track key pollutants, such as PM2.5, PM10, CO, and CO2, to detect and mitigate poor ventilation. The system analyzes data through the k-Nearest Neighbors (k-NN) algorithm and achieved impressive results with a precision of 0.91 and an F1-score of 0.89, showcasing its capability in accurately detecting inadequate ventilation. This high level of accuracy allows the system to issue real-time alerts through a smartphone app, enabling immediate corrective actions to enhance air quality. The study demonstrates the effectiveness of data-driven approaches in improving ventilation control, and Rastogi et al. proposed incorporating additional environmental parameters such as light, sound, and occupancy into the system. This would enhance the adaptability and accuracy of the system, making it more effective across various environments with different air quality challenges \cite{2}.

Mobasshir Mahbub et al. developed an automated system for controlling air conditioning and lighting based on occupancy sensors, alongside monitoring other environmental factors like temperature, humidity, CO2, and smoke. The system ensures proper ventilation and fire safety while optimizing energy usage. It allows for remote monitoring and control via mobile phones or computers, offering a robust solution for real-time management of indoor air quality. By storing data in a cloud database, the system provides convenient access for users to monitor conditions remotely. The authors also discussed potential future improvements, such as integrating machine learning for predictive fire detection and enhancing cloud-based visualizations to facilitate deployment in industrial environments or densely populated areas. The ability to integrate smart technologies into existing building management systems could lead to significant advances in indoor air quality management and safety in various settings \cite{3}.

Tagliabue et al. proposed an IoT-based system that integrates indoor air quality (IAQ) data with HVAC (Heating, Ventilation, and Air Conditioning) control systems, aiming to enhance user comfort and performance in educational environments. The system collects real-time data on CO2 levels, temperature, and humidity, forecasting comfort conditions and adjusting ventilation rates using artificial neural networks (ANNs) and Markov predictive models. The results from their system showed reliable predictions, with CO2 forecasts achieving a mean squared error (MSE) of 10.6\%, proving its potential to adjust ventilation dynamically to optimize comfort and air quality. Future work could refine the system by integrating additional parameters, such as occupancy counts, room activity, and user feedback, to improve prediction accuracy and overall system performance. The incorporation of real-time data from users would also enhance the personalization of air quality management in educational settings, benefiting both students and staff \cite{4}.

Wang et al. introduced a novel CT-LSTM (Convolutional Transformer Long Short-Term Memory) model for predicting Air Quality Index (AQI) levels, using both air quality and meteorological data to improve prediction accuracy. By leveraging the powerful capabilities of LSTM networks, the model captures sequential dependencies in air quality data, outperforming traditional predictive models like Support Vector Regression (SVR), Multi-Layer Perceptron (MLP), and basic Recurrent Neural Networks (RNNs). Their model demonstrated superior prediction accuracy, suggesting its applicability to other time-series forecasting problems such as stock market trends and commodity pricing. Wang et al. noted that their model's ability to predict AQI could be a critical tool in public health and environmental monitoring. Future research could focus on enhancing the model with multi-scale spatial predictions, allowing for improved accuracy and broader applicability in various geographical environments \cite{5}.

Yang et al. developed a hybrid BO–EMD–LSTM (Bayesian Optimization-Empirical Mode Decomposition-Long Short-Term Memory) method for predicting indoor CO2 levels, a significant factor in indoor air quality management. This method combines signal decomposition with deep learning and Bayesian optimization techniques, achieving more than 55\% higher accuracy than traditional models, with an R² value exceeding 95\%, which indicates excellent prediction stability. Yang et al. emphasized the speed and high accuracy of the model, making it ideal for real-time applications such as indoor air quality monitoring systems. The use of such advanced techniques allows for dynamic, real-time adjustments to air quality management systems. Future work could explore how subsequence analysis can enhance prediction performance, particularly in scenarios with fluctuating environmental conditions. Furthermore, enhancing the computational efficiency of the model could enable its widespread deployment in various smart building applications \cite{6}.

McNeill et al. investigated ventilation rates across different schools and universities to assess their role in reducing the transmission of airborne diseases. Their study showed that adequate ventilation, particularly through mechanical ventilation systems, was crucial for improving air quality and minimizing disease transmission in classrooms. The research revealed that classrooms with mechanical ventilation had air change rates (ACH) ranging from 2.7 to 8.7, while naturally ventilated classrooms often had ACH values below the recommended minimum of 3 ACH. The study also found that rooms with open windows could achieve an ACH of up to 21.3, surpassing the World Health Organization (WHO) guidelines for healthcare settings. Their work stresses the importance of effective ventilation in educational spaces, particularly in light of the ongoing global health challenges. McNeill et al. called for further research into cost-effective, real-time monitoring solutions, particularly for naturally ventilated spaces, to enhance public health safety \cite{7}.

Dhanalakshmi et al. proposed an IoT-based approach to managing indoor air quality (IAQ) and optimizing energy usage in HVAC systems. Their system dynamically adjusts HVAC operations based on CO2 and temperature readings while integrating user feedback to maintain comfort and achieve significant energy savings. The system demonstrated an annual energy savings of 328.5 kWh, illustrating the potential for energy-efficient ventilation solutions in buildings. By automating ventilation and air conditioning based on real-time data, the system can ensure energy-efficient operation without compromising indoor comfort. Dhanalakshmi et al. suggested that future research could explore the integration of machine learning techniques to further improve predictive accuracy and optimize HVAC operations based on historical and real-time data \cite{8}.

Fayos-Jordan et al. introduced VentQsys, an IoT-based open system designed to enhance classroom ventilation by continuously monitoring CO2 levels. The system utilizes low-cost, Wi-Fi-enabled sensors to monitor CO2, temperature, and humidity, allowing real-time adjustments to ventilation. This continuous monitoring improves air quality and reduces health risks associated with high CO2 concentrations. Fayos-Jordan et al. emphasized the importance of integrating 5G technology into the system to enable faster data transfer, which could enhance the responsiveness of the ventilation system. Additionally, the authors highlighted the potential of ultra-low-power processors to extend battery life, which is critical for long-term, low-maintenance applications in educational settings. The use of such technology could be instrumental in ensuring better air quality management in schools and universities \cite{9}.

Rizzo et al. developed an IoT-based demand-controlled ventilation (DCV) system aimed at optimizing indoor air quality (IAQ) and thermal comfort in classrooms. By integrating CO2, temperature, and humidity sensors with a building management system (BMS), the system dynamically adjusts ventilation based on CO2 concentrations. This results in improved air quality while minimizing energy losses by prioritizing the intake of preheated air from corridors during colder months. Rizzo et al. suggested that future advancements could include incorporating variable refrigerant flow (VRF) systems to further enhance energy efficiency and maintain thermal comfort in classrooms. This system could be particularly valuable for reducing heating and cooling costs while ensuring optimal indoor air quality \cite{10}.

Zivelonghi et al. presented AulaSicura, an IoT-based system developed under the Smart Healthy Schools (SHS) initiative to enhance classroom ventilation through continuous CO2 monitoring. The system uses LoRaWAN-enabled sensors to monitor CO2, temperature, and humidity, enabling centralized control of ventilation systems to adjust based on real-time data. This approach minimizes energy loss by dynamically adjusting ventilation methods based on the monitored conditions. Future developments could include predictive optimization algorithms that would further improve energy efficiency while ensuring that IAQ standards are met, leading to healthier and more sustainable learning environments \cite{11}.

Idrees et al. proposed an edge-computing architecture for real-time air quality and ventilation monitoring in enclosed spaces. Their IoT-based system integrates low-cost pollutant sensors (PM2.5, CO, SO2) alongside temperature and humidity sensors. These sensors send data to an edge device for processing, which reduces the burden on cloud resources and enables faster decision-making. The decentralized processing architecture improves response times and ensures data privacy by keeping sensitive information on local devices. Idrees et al. also noted that future work could enhance the system by incorporating more advanced sensors and improving sensor calibration to boost data accuracy, thereby enhancing the system's overall reliability and performance \cite{12}.

Marques et al. introduced 'iDust,' an IoT-based system designed for real-time particle monitoring to improve indoor air quality. The system uses a WEMOS D1 mini microcontroller and PMS5003 sensor to measure particulate matter (PM1.0, PM2.5, and PM10) levels, displaying the results on a web-based dashboard. This low-cost and scalable solution is ideal for building management systems and can easily be adapted for use in hospitals and other healthcare environments. Marques et al. emphasized the importance of ensuring data security, particularly in medical environments, and suggested that future work should focus on enhancing the system's security features to meet compliance standards for medical diagnostics \cite{13}.

El-Sayed et al. explored the potential of augmented reality (AR) in combination with mobile edge computing (MEC) to optimize energy consumption and resource allocation. Their research highlighted how MEC can reduce energy consumption by processing data closer to the user, thus reducing the load on central servers and enhancing AR application performance. By analyzing communication, computational, and energy consumption metrics, the study demonstrated that MEC could optimize resources in AR settings, leading to better user experience and lower energy costs. El-Sayed et al. suggested that future research could focus on refining these methods to further reduce energy consumption, particularly in mobile and IoT contexts, where energy efficiency is critical \cite{14}.

Jagriti Saini et al. presented a dataset aimed at optimizing dynamic resource allocation in edge computing (EC) for IoT applications. Their dataset focuses on resource sharing between cloud and edge computing environments, which is essential for optimizing IoT systems. By analyzing metrics like latency, bandwidth, and resource availability, the authors demonstrated the importance of bidirectional resource management to enhance IoT network performance. Future research could explore machine learning techniques to enable real-time resource allocation, further optimizing overall system efficiency \cite{15}.

Mohieddine Benammar et al. introduced a dataset for the CooLoad load balancing scheme, focusing on edge network buffer management in distributed data centers. The dataset tracks buffer space availability, optimizing data flow and reducing delays. This dynamic load distribution approach enhances network performance, especially for latency-sensitive IoT applications. The study provides insights into buffer management strategies and suggests that predictive load balancing could further improve data handling in edge computing environments \cite{16}.

Jung-Yoon Kim et al. analyzed wireless software-defined mobile edge computing (SDMEC) frameworks with a focus on auto-scaling for storage management in edge networks. Their dataset demonstrated how storage capacity could be efficiently managed through auto-scaling techniques, reducing latency and improving the quality of experience (QoE) for users. The study emphasizes efficient resource allocation to ensure optimal performance in edge networks. Future work could focus on integrating predictive models to further optimize storage allocation during periods of peak demand \cite{17}.


Mehmet Taştan and Hayrettin Göközan (2019)
Developed an IoT-based electronic nose (e-nose) system using metal-oxide gas sensors and machine learning to monitor indoor air quality in real time. The system detected volatile organic compounds (VOCs) and CO₂ with 92\% accuracy, leveraging edge computing for rapid anomaly detection. Future work could integrate predictive maintenance algorithms to preempt sensor drift in long-term deployments\cite{18}.

 Kenichi Azuma et al. (2018)
Reviewed low-level CO₂ exposure (500–5,000 ppm) effects, showing linear increases in heart rate and blood pressure, along with impaired decision-making at ≥1,000 ppm. The study highlighted correlations between chronic exposure >700 ppm and respiratory symptoms in children, though confounding pollutants (e.g., VOCs) complicate isolation of CO₂ impacts. Future research should investigate synergistic effects of CO₂ with other indoor pollutants \cite{19}.

 Xilei Dai et al. (2023)
Proposed an IoT architecture combining multi-sensor networks (CO₂, PM2.5, humidity) with edge-based federated learning to optimize indoor air quality (IAQ) while preserving data privacy. Challenges included interoperability across legacy HVAC systems and energy efficiency trade-offs. Opportunities involve integrating digital twins for scenario simulation. \cite{20}.

 J. Chen et al. (2022)
Applied transfer learning to predict PM₂.₅ levels across cities, training models on data-rich source cities (Beijing, Delhi) and fine-tuning on target cities (Mumbai, Bangkok) with 30\% less data. The approach reduced prediction errors by 18\% compared to local models, demonstrating viability for CO₂ generalization. Future work could explore domain adaptation for varying urban topographies \cite{21}.

 S. Wu et al. (2022)
Achieved 90\% accuracy in detecting patient–ventilator asynchrony (PVA) using transfer learning with pretrained ResNet-50 models on 1D respiratory signals converted to spectrograms. The system required only 1\% of annotated data, enabling edge deployment on Raspberry Pi 4. Limitations included latency in real-time inference, suggesting future optimizations \cite{22}.

 Y. Wang et al. (2023)
Implemented a dynamic mobile window mechanism on edge devices to adaptively update LSTM models for CO₂ prediction, reducing mean absolute error (MAE) from 58 ppm to 42 ppm over 30 days. The system prioritized recent data through a sliding window, improving responsiveness to occupancy changes. Future integration with building digital twins is proposed \cite{23}.

 Zhang et al. (2021)
Deployed an LSTM model on ESP32 microcontrollers for steady-state CO₂ forecasting, achieving 5.5\% error using MQTT-based sensor data. The edge-computed predictions enabled proactive ventilation control in office spaces, reducing energy use by 12\%. Future work could apply transfer learning to generalize across building types \cite{24}.

 L. Sun et al. (2022)
Designed a model predictive control (MPC) system on Raspberry Pi 4B to optimize CO₂ levels (<800 ppm) and energy consumption in real time. The edge-deployed solution reduced peak HVAC loads by 22\% through occupancy-driven ventilation. Future directions include hybrid physics-AI models for unknown environments \cite{25}.
