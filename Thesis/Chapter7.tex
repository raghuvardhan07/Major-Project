
\section{BIBLIOGRAPHY}
\begin{thebibliography}{00}
\bibitem{b2} 
\bibitem{b3} 
\bibitem{b4} 
\bibitem{b5} 
\bibitem{b6} 
\bibitem{b7}
\bibitem{b8} 
\bibitem{b2} WiDS Datathon 2021 Data. Accessed: Jan. 6, 2021. [Online]. Available:
https://www.kaggle.com/c/widsdatathon2021/data
\bibitem{b2}J. E. Zimmerman, A. A. Kramer, D. S. McNair, and F. M. Malila, ‘‘Acute
physiology and chronic health evaluation (APACHE) IV: Hospital mortality assessment for today’s critically ill patients,’’ Crit. Care Med., vol. 34,
no. 5, pp. 1297–1310, May 2006.
\bibitem{b2}  M. Lee, J. Raffa, M. Ghassemi, T. Pollard, S. Kalanidhi, O. Badawi,
K. Matthys, and L. A. Celi, ‘‘WiDS (women in data science) datathon
2020: ICU mortality prediction (version 1.0.0),’’ PhysioNet, 2020,
doi: 10.13026/vc0e-th79.
\bibitem{b2}J. Raffa, A. Johnson, L. A. Celi, T. Pollard, D. Pilcher, and O. Badawi,
‘‘33: The global open source severity of illness score (GOSSIS),’’ Crit.
Care Med., vol. 47, no. 1, p. 17, 2019.
\bibitem{b2}M. Michailidis, ‘‘Investigating machine learning methods in recommender
systems,’’ Ph.D. dissertation, Dept. Financial Comput., Univ. College London, London, U.K., 2017.
\bibitem{b2} A. Krizhevsky, I. Sutskever, and E. G. Hinton, ‘‘Imagenet classification
with deep convolutional neural networks,’’ in Proc. Adv. Neural Inf. Process. Syst., 2012, 1097-1105.
\bibitem{b2}C. Shorten and T. M. Khoshgoftaar, ‘‘A survey on image data augmentation
for deep learning,’’ J. Big Data, vol. 6, no. 1, p. 60, Dec. 2019.
\bibitem{b2}F. Perez, C. Vasconcelos, S. Avila, and E. Valle, ‘‘Data augmentation for
skin lesion analysis,’’ in OR 2.0 Context-Aware Operating Theaters, Computer Assisted Robotic Endoscopy, Clinical Image-Based Procedures, and
Skin Image Analysis. Cham, Switzerland: Springer, 2018, pp. 303–311.
\bibitem{b2} J. S. Friedland, J. C. Porter, S. Daryanani, J. M. Bland, N. J. Screaton,
M. J. J. Vesely, G. E. Griffin, E. D. Bennett, and D. G. Remick, ‘‘Plasma
proinflammatory cytokine concentrations, acute physiology and chronic
health evaluation (APACHE) III scores and survival in patients in an
intensive care unit,’’ Crit. Care Med., vol. 24, no. 11, pp. 1775–1781,
Nov. 1996.
\bibitem{b2} J. R. Le Gall, ‘‘A new simplified acute physiology score (SAPS II) based
on a European/North American multicenter study,’’ JAMA: J. Amer. Med.
Assoc., vol. 270, no. 24, pp. 2957–2963, Dec. 1993.
\bibitem{b2} S. Lemeshow, ‘‘Mortality probability models (MPM II) based on an international cohort of intensive care unit patients,’’ JAMA: J. Amer. Med.
Assoc., vol. 270, no. 20, pp. 2478–2486, Nov. 1993.
\bibitem{b2} S. F. B. Zaidi, M. A. Raouf, and T. Tariq, ‘‘Comparison of APACHE II,
SAPS II and SOFA scoring systems as predictors of mortality in ICU
patients,’’ Pervaiz Ellahi Inst. Cardiol., Multan, Pakistan, Tech. Rep., 2019,
vol. 53, doi: 10.7176/JMPB.
\bibitem{b2} T. A. Ayazoglu, ‘‘A comparison of APACHE II and APACHE IV scoring
systems in predicting outcome in patients admitted with stroke to an
intensive care unit,’’ Anaesthesia, Pain & Intensive Care, vol. 15, no. 1,
pp. 7–12, 2019.
\bibitem{b2}R. Lior, ‘‘Ensemble-based classifiers,’’ Artif. Intell. Rev., vol. 33, nos. 1–2,
pp. 1–39, 2010.
\bibitem{b2}A. P. Bradley, ‘‘The use of the area under the ROC curve in the evaluation of machine learning algorithms,’’ Pattern Recognit., vol. 30, no. 7,
pp. 1145–1159, Jul. 1997.
\bibitem{b2} WiDS Datathon 2020 Data. Accessed: Jan. 11, 2020. [Online]. Available:
https://www.kaggle.com/c/widsdatathon2020/data
\bibitem{b2}J. H. Friedman, ‘‘Greedy function approximation: A gradient boosting
machine,’’ in Annals of Statistics. Stanford, CA, USA: Stanford Univ.,
2001, pp. 1189–1232.
\bibitem{b2} L. Breiman, ‘‘Random forests,’’ Mach. Learn., vol. 45, no. 1, pp. 5–32,
2001.
