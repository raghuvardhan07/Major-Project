\chapter{System Requirement Specification}

This section outlines the essential requirements for designing, developing, and deploying the intelligent classroom ventilation system. The specifications are categorized into functional, non-functional, software, and hardware requirements to ensure the system performs efficiently, remains user-friendly, and delivers accurate, real-time predictions for classroom air quality and occupancy. These requirements provide the foundation for creating a reliable, scalable, and intelligent edge-based monitoring solution suitable for educational environments.

\section{Functional Requirements}
This system is designed with one clear goal in mind: to make classrooms healthier and more comfortable by keeping an eye on environmental conditions in real time. At the heart of the system are two Arduino-based units installed in separate classrooms. Each of these units collects important data such as carbon dioxide (CO₂) levels, temperature, humidity, and how many people are currently in the room. The people count is tracked using IR sensors that detect when someone enters or exits the classroom. All of this sensor data is sent to a central Raspberry Pi, which does the heavy lifting. It runs a specially trained machine learning model (an LSTM model) that can predict air quality and calculate something called the KPIv—a custom score we created to evaluate how well-ventilated a room is. Based on this, the system can figure out which classroom is in better shape to be occupied at any given moment. It doesn't stop at just predictions. If the air quality drops, the system kicks into action: CPU fans switch on to improve ventilation, buzzers sound alarms, and LEDs light up to alert the room occupants. Every few minutes (usually between 3 to 5), the system logs all its readings and pushes them to the cloud using ThingSpeak. This gives us a centralized place to store and visualize data, and it also powers the automatic generation of certificates based on how well the classroom is performing over time. To make things easy for anyone using the system, each classroom has its own LCD display that shows the latest temperature, humidity, and air quality status. The Raspberry Pi and Arduino are in constant communication, which allows real-time updates and makes the system smart enough to adapt to changing conditions quickly. Overall, the system is built to work quietly in the background, making decisions and adjustments on its own to maintain the best environment possible for learning.
\\
\section{Non-functional Requirements}
For ensuring reliability, efficiency, & operational excellence, several non-functional requirements need to be fulfilled. The system must be able to perform tasks such as environmental monitoring & ventilation control within two seconds of detection. It's essential that the system is optimized for rapid response. In this manner, classroom conditions remain healthy with minimal delay in environmental control. The monitoring system must maintain continuous operation during school hours. This is a minimum expectation for how well it must perform. To keep system reliability high is essential, as it makes environmental control dependable. It's essential for making critical ventilation decisions. Perhaps, the system must incorporate scalability features. It must be able to handle monitoring additional classrooms without reducing its performance. The implementation utilizes a distributed architecture with a central Raspberry Pi controller. This facilitates efficient processing of sensor data as necessary for multiple classrooms. Security features are required as well. They must involve proper electrical isolation & procedures for protecting sensor connections, using appropriate voltage regulation. The system's design is user-friendly. This is where staff can interpret alerts, understand environmental conditions, & respond to warnings easily, without special technical skills. Finally, durability is essential. The design must provide stable operation even where classroom conditions vary or where there are different levels of occupancy & activity.

\section{Software Requirements}\\
On the software side of things, we’re working with a combination of Python and C a mix that brings together flexibility, speed, and control. The Raspberry Pi runs the Python code, which is responsible for loading the trained LSTM model, processing incoming sensor data, and handling communication with the cloud. Python libraries like TensorFlow (for machine learning), NumPy and Pandas (for data manipulation), and Scikit-learn (for extra ML tools) are all part of the stack. We also use Matplotlib to visualize trends and debug during development. Meanwhile, the Arduinos in each classroom run C code written in the Arduino IDE. They handle all the direct interaction with sensors and hardware reading data from DHT11 and CO₂ sensors, detecting movement via IR sensors, updating the LCD displays, and managing the buzzers and LEDs. They also handle serial communication with the Raspberry Pi, acting as the hands and ears of the system, while the Pi is the brain. To upload and visualize the collected data, we rely on ThingSpeak, which gives us a nice dashboard for monitoring everything and also plays a key role in the automatic generation of environmental certificates. These certificates are issued based on how well a classroom maintains good air quality, making the system not just practical, but also rewarding. We typically write and test the Python code using editors like Thonny or Visual Studio Code. The software is modular and well-documented, so it's easy to make changes if we want to update the model or tweak how often data gets logged. Everything has been built with real-time performance, security, and future improvements in mind.



\section{Hardware Requirements}
The system requires appropriate hardware components for running environmental monitoring effectively. We utilize two Arduino Uno boards as edge devices, one for each classroom, handling sensor interfacing \& initial data processing. For central processing, a Raspberry Pi 4 with 4GB RAM serves as the fog layer controller, efficiently managing the LSTM model \& communication protocols. For environmental sensing, each classroom unit employs multiple critical sensors. An MQ135 sensor handles CO2 monitoring with detection ranges from 400ppm to 2000ppm. The DHT11 sensor provides temperature and humidity. Occupancy monitoring is implemented through paired IR sensors at doorways, enabling bidirectional people counting. Display functionality requires 16x2 LCD screens for each classroom unit, connected through GPIO pins for real-time status display. The ventilation control system employs 12V DC fans, controlled through relay modules with optocoupler isolation for safe operation. Power management is another critical aspect. There must be stable power supplies - 12V for fans and 5V/3A for the Raspberry Pi, with appropriate voltage regulation and filtering. Serial communication between Arduino units and Raspberry Pi operates at 9600 baud rate through USB ports, ensuring reliable data transmission. Also, there must be proper wiring and connections. This facilitates smooth data transfer as well as enables real-time system response. After implementing the above-mentioned hardware components, the system can actually deliver excellent environmental monitoring while performing efficiently as well as maintaining reliable operation. When the air quality drops, red LEDs light up and buzzers make a sound to alert everyone. If needed, the Raspberry Pi also tells relay modules to switch on CPU fans, helping to bring the air quality back to normal. Overall, the hardware has been carefully selected to balance performance, cost, and expandability, creating a robust and intelligent system that actively improves the air quality in learning spaces.




