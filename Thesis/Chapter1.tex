\chapter{Introduction}

\section{Origin of Proposal}
Indoor air quality significantly impacts cognitive ability, comfort, and overall health, especially in tightly occupied areas such as classrooms. Traditionally, indoor ventilation management has relied on static, threshold-based systems or manual operations that lack responsiveness to real-time environmental changes. These approaches are often inefficient, unable to account for varying occupancy levels or fluctuating CO\textsubscript{2} concentrations, and offer limited adaptability in dynamic learning environments.Existing monitoring systems fail to integrate predictive modeling with localized control, leading to inconsistent ventilation performance and energy inefficiencies. The challenge lies in developing a responsive system capable of collecting real-time data, interpreting temporal patterns, and adjusting ventilation intelligently. Advancements in edge computing and deep learning, particularly through Long Short-Term Memory (LSTM) networks, provide new opportunities for accurate forecasting of indoor conditions.

This proposal arises from the need for a robust, predictive, and automated classroom ventilation solution. It seeks to ensure healthy air quality, sustainable energy usage, and real-time certification of environmental standards, contributing to smarter and healthier educational infrastructure.

\section{Definition of Problem}
Accurate prediction and control of indoor CO\textsubscript{2}   levels in educational spaces are vital for improving cognitive performance, maintaining occupant comfort, and ensuring a healthy learning environment. However, several challenges hinder effective ventilation optimization. Environmental data from classrooms often display fluctuations due to changing occupancy levels, variable room sizes, and diverse usage patterns, complicating the task of consistent air quality regulation. Additionally, the latency and limitations of cloud-based systems in handling real-time control reduce their practicality in fast-changing indoor environments. Edge devices, while beneficial for localized processing, often face resource constraints when dealing with continuous data streams and advanced predictive models. Furthermore, the nonlinear and temporal nature of CO₂ progression demands models that can learn from sequential dependencies rather than static thresholds. Traditional rule-based or shallow learning approaches struggle to capture these complex dynamics. To overcome these limitations, our system integrates a lightweight Long Short-Term Memory (LSTM) model with sensor networks and actuator control, all deployed on edge platforms. This architecture enables timely, adaptive ventilation management with minimal computational overhead while enhancing responsiveness and prediction accuracy across diverse classroom conditions.

\\ 

\section{Objective}
The primary objective of this project is to design, implement, and evaluate a predictive system for real-time classroom air quality management. The goal is to enhance the efficiency, comfort, and sustainability of indoor environments by using machine learning techniques and edge computing to predict air quality and optimize ventilation control. The specific goals are elaborated below:

To develop predictive models using Long Short-Term Memory (LSTM) networks and regression algorithms for accurate CO₂ level forecasting. This project proposes an approach that combines LSTM's ability to capture temporal dependencies in environmental data with regression techniques to refine ventilation control. The motivation behind this design is to forecast future air quality trends, enabling the system to proactively adjust ventilation instead of reacting to changes. The model is expected to improve air quality prediction and ventilation control, adapting to various classroom conditions. To validate its effectiveness, the LSTM-based model will be compared to traditional threshold-based methods and other machine learning approaches.

To implement a real-time data acquisition and monitoring system using low-cost IoT sensors and edge devices. The system will integrate environmental sensors for CO₂, temperature, humidity, and occupancy, all connected to platforms like Raspberry Pi and Arduino. This architecture allows for efficient, low-latency data processing, ensuring real-time air quality assessment while minimizing dependence on cloud infrastructure. The edge-based setup enhances data privacy, speeds up response times, and offers scalability for deployment in multiple classrooms or institutions.

To investigate the influence of key environmental variables on air quality and occupancy patterns within classrooms. This research will identify which factors—such as CO₂ concentration, temperature, and occupancy—affect indoor air quality the most. By recognizing these patterns, the system can better adjust ventilation strategies to improve air quality, ensuring a comfortable learning environment. This will lead to optimized ventilation systems and more effective classroom management, promoting student and faculty health.

To create a decision support tool that recommends the best classroom based on real-time environmental conditions. This tool will assess classroom environments in real time and suggest the optimal room with the best air quality for students and faculty. By doing so, it will help in selecting classrooms that meet air quality standards, ultimately improving learning conditions.

To integrate the predictive air quality system with classroom management tools and cloud platforms for seamless monitoring and data processing. This integration will enable automated air quality tracking, real-time data access, and long-term monitoring, contributing to the development of smarter, healthier classroom environments and supporting the growth of sustainable educational infrastructure.
\newpage
\section{Organization Of the Project}\\
Chapter 1: This introductory chapter provides background information on the importance of classroom air quality and the need for an intelligent ventilation system. It discusses the project's goals, the integration of machine learning for air quality prediction, and the benefits of real-time monitoring in educational environments.

Chapter 2: This chapter reviews existing methods for air quality management and smart classroom systems, with a focus on the use of predictive models like LSTM for environmental forecasting. It also discusses IoT-based data acquisition systems and their role in improving classroom air quality and comfort.

Chapter 3: This chapter outlines the system requirements, both functional and non-functional, for implementing the air quality management system. It details the hardware and software components necessary to deploy the predictive model and the real-time monitoring system.

Chapter 4: The data collection and preprocessing methods used in this project are discussed in this chapter, along with the structure of the predictive model. It describes how environmental sensor data is gathered and how the LSTM model is applied to predict CO₂ levels and inform ventilation control decisions.

Chapter 5: This chapter covers the implementation of the methods, including data handling, model training, and performance evaluation. It discusses the evaluation metrics used, such as accuracy, CO₂ level prediction error, and system responsiveness. A comparative analysis of the model's performance is also presented.

Chapter 6: This chapter presents the findings of the study, highlighting the effectiveness of the predictive model and its impact on air quality management. It also discusses the practical applications of the system for real-time classroom air quality control.

Chapter 7: The final chapter explores potential future enhancements, including advanced data augmentation techniques, real-time adaptation to classroom conditions for large-scale deployment in educational institutions.