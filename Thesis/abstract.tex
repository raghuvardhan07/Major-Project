\clearpage
\phantomsection
\addcontentsline{toc}{chapter}{Abstract}
\chapter*{Abstract}
As indoor air quality (IAQ) becomes increasingly critical in educational settings, optimizing classroom ventilation systems is essential for enhancing student and teacher well-being. This study presents an innovative edge computing-based approach that combines IoT sensors, LSTM neural networks, and cloud integration to create an intelligent ventilation management system. By deploying low-cost IoT devices equipped with environmental sensors and edge computing capabilities, our solution collects and processes data at the classroom level. A lightweight LSTM model implemented on edge devices provides real-time predictions of ventilation quality and occupancy patterns, enabling immediate environmental adjustments. The system architecture features a hierarchical design that includes local edge devices for data acquisition and initial predictions, an intermediate fog layer utilizing Raspberry Pi for aggregation and building-wide decision-making, and cloud services (ThingSpeak) for long-term data storage and analysis. The real-time data processing with minimal latency, and comprehensive environmental quality certificates. The system successfully monitors and analyzes key parameters including temperature, humidity, CO2 levels, and occupancy rates, demonstrating its effectiveness in maintaining optimal classroom conditions. The integration of edge computing with machine learning and IoT technologies delivers improved precision in environmental quality evaluation, making it highly useful for educational institutions.
\\
\\

\textbf{\textit{Keywords}}:
Smart Classroom, Environmental Quality, CO\textsubscript{2}
 Prediction, LSTM, Edge Computing, Occupancy Monitoring, Ventilation Control, Certificate Generation

